\section{Auswertung}
\label{sec:Auswertung}

Zunächst werden die Grenzspannungen $U_G$ der jeweiligen Spektrallinien bestimmt.
Dafür werden in \autoref{fig:spek1} bis \autoref{fig:spek7} die Gegenspannungen gegen die Wurzel des Stroms aufgetragen.
Die jeweiligen Messwerte dazu sind in den Tabellen \ref{tab:spek1} - \ref{tab:spek7} zu finden.
Die Ausgleichsgerade wird über den Ansatz 
\begin{equation*}
  U_G = - \frac{a}{b}
\end{equation*}
berechnet.

\begin{tabular}{c c c}
  \toprule
   U / V &  I / 10\textasciicircum -9 A &  sqrt(I) / 10\textasciicircum -5 A\textasciicircum (1/2) \\
  \midrule
       1 &            1 &                        1 \\
       1 &            1 &                        1 \\
  \bottomrule
\end{tabular}