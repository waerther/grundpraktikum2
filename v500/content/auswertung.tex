\section{Auswertung}
\label{sec:Auswertung}

Zunächst werden die Grenzspannungen $U_G$ der jeweiligen Spektrallinien bestimmt.
Dafür werden in \autoref{fig:spek1} bis \autoref{fig:spek5} die Gegenspannungen gegen die Wurzel des Stroms aufgetragen.
Die jeweiligen Messwerte dazu sind in den Tabellen \ref{tab:spek1} - \ref{tab:spek5} zu finden.
Die Ausgleichsgerade wird über den Ansatz 
\begin{equation*}
  U_G = - \frac{a}{b}
\end{equation*}
berechnet.

\begin{table}
  \centering
  \caption{Messwerte der ultravioletten Linie.}
  \label{tab:spek1}
  \begin{tabular}{c c}
    \toprule
    U / V & I / 10\textasciicircum -9 A \\
    \midrule
    1.283 &         0,0 \\
      1,2 &        0,04 \\
      1,1 &       0,075 \\
      1,0 &       0,145 \\
      0,9 &         0,2 \\
      0,8 &       0,298 \\
      0,7 &       0,415 \\
      0,6 &       0,540 \\
      0,5 &       0,660 \\
      0,4 &       0,780 \\
      0,3 &       0,920 \\
      0,2 &         1,0 \\
      0,1 &         1,2 \\
     0,05 &         1,2 \\
     0,00 &         1,3 \\
    \bottomrule
  \end{tabular}
\end{table}

\begin{table} 
  \centering
  \caption{Messwerte der violetten Linie.}
  \label{tab:spek2}
  \begin{tabular}{c c}
    \toprule
    U / V & I / 10\textasciicircum -9 A \\
    \midrule
    1,152 &         0,0 \\
      1,1 &        0,08 \\
      1,0 &        0,34 \\
      0,9 &        0,72 \\
      0,8 &         1,2 \\
      0,7 &         2,0 \\
      0,6 &         2,9 \\
      0,5 &         4,0 \\
      0,4 &         5,0 \\
      0,3 &         5,9 \\
      0,2 &         6,8 \\
      0,1 &         7,0 \\
     0,05 &         8,2 \\
     0,00 &         8,7 \\
    \bottomrule
    \end{tabular}
\end{table}

\begin{table}
  \centering
  \caption{Messwerte der grünen Linie.}
  \label{tab:spek3}
  \begin{tabular}{c c}
    \toprule
    U / V & I / nA \\
    \midrule
    0,645 &    0,0 \\
      0,6 &   0,12 \\
     0,55 &  0,295 \\
     0,50 &   0,56 \\
     0,45 &   0,62 \\
     0,40 &    1,4 \\
     0,35 &    2,0 \\
      0,3 &    2,8 \\
     0,25 &    3,5 \\
      0,2 &    4,2 \\
     0,15 &    4,8 \\
      0,1 &    5,3 \\
     0,05 &    5,8 \\
      0,0 &    6,2 \\
    \bottomrule
    \end{tabular}
\end{table}

\begin{table}
  \centering
  \caption{Messwerte der gelben Linie.}
  \label{tab:spek4}
  \begin{tabular}{c c c c}
    \toprule
    U / V & I / nA & U / V.1 & U / nA \\
    \midrule
     0,52 &    0,0 &   -0,05 &    2,4 \\
      0,5 &   0,02 &    -0,1 &    2,6 \\
     0,45 &    0,1 &    -0,5 &    3,8 \\
      0,4 &   0,22 &    -1,0 &    5,4 \\
     0,35 &    0,4 &   - 1,5 &    6,4 \\
      0,3 &   0,63 &    -2,0 &    7,6 \\
     0,25 &   0,96 &    -3,0 &   10,0 \\
      0,2 &    1,2 &    -4,0 &   12,0 \\
     0,15 &    1,5 &    -6,0 &   12,0 \\
      0,1 &    1,7 &   -10,0 &   17,0 \\
     0,05 &    2,0 &   -14,0 &   20,0 \\
      0,0 &    2,2 &   -18,0 &   20,0 \\
      NaN &    NaN &    19,0 &   21,0 \\
    \bottomrule
  \end{tabular}
\end{table}

\begin{table}
  \centering
  \caption{Messwerte der roten Linie.}
  \label{tab:spek5}
  \begin{tabular}{c c}
    \toprule
    U / V & I / nA \\
    \midrule
      0,5 &    0,0 \\
     0,45 &  0,005 \\
      0,4 & 0,0075 \\
     0,35 &   0,02 \\
      0,3 &  0,021 \\
     0,25 &  0,035 \\
      0,2 &   0,04 \\
     0,15 &   0,05 \\
      0,1 &   0,06 \\
     0,05 &  0,075 \\
     0,00 &   0,08 \\
    \bottomrule
    \end{tabular}
\end{table}