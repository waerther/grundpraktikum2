\section{Diskussion}
\label{sec:Diskussion}

Die Regressionsgeraden, dargestellt in \autoref{fig:spek1} bis \autoref{fig:spek5}, sind jeweils
nah an den jeweiligen Messwerten, weshalb die Messung der einzelnen Grenzspannungen eigentlich als Erfolg gewertet werden könnte.
Jedoch gibt es in der folgenden Auswertung zum Verhältnis von $\frac{h}{e_0}$ große Abweichungen zum Literaturwert,
der sich auf $4,136 \cdot 10^{-15} \unit\eV$ beläuft.
Damit ergibt sich eine Abweichung von 
\begin{equation*}
    \increment \frac{h}{e_0} = \frac{274 - 4.145}{4.135} \cdot 100 = 6526.36 \%.
\end{equation*}
Das ist eine deutliche Abweichung. Jedoch konnte selbst nach einer intensiven Nacharbeitung kein systematischer Fehler entdeckt werden.
Die Austrittsarbeit wurde zu $A_K = (0.78 ±  0.22) \unit{\eV}$ ermittelt.

Die Ergebnisse von der dritten Messung bedarf noch einer intensiven Diskussion.
Zunächst ist ein Sättigungswert erwartbar, da ab einer gewissen Spannung $U$ annähernd alle
Elektronen die ausgelöst werden auch an die Anode gelangen.
Diese Grenze ist durch die Intensität des Lichtes festgelegt, weshalb diese bei der Durchführung 
konstant war. Jedoch darf der Einfluss des äußeren Tageslichtes auf den Versuch nicht vergessen werden.
Das Tageslicht war bei diesem Versuch vermutlich die größte Fehlerquelle.
Ist dieser Zusammenhang erkannt worden, ergibt sich auch kein Widerspruch zum Ohm'schen
Gesetz, welches als
\begin{equation*}
    I = \frac{U}{R}
\end{equation*}
bekannt ist, da dieser Effekt unabhängig von diesem Gesetz ist.
Der Sättigungswert kann hier aufgrund der technischen Realisierbarkeit nur asymptotisch erreicht werden,
da dafür mehr als $19 \unit\volt$ als Beschleunigungsspannung nötig sind.
Wollte man jedes Elektron im Strom messen, müsste man U gegen unendlich schicken.
Das ist nicht realisierbar.
Es sollte außerdem die Frage geklärt werden, warum der Anodenstrom bereits vor der
Grenzspannung abnimmt. Dies liegt an der Verteilung der Elektronen im Metall (Fermi-Dirac-Statistik).
Aufgrund dieser Verteilung besitzen die Elektronen unterschiedliche kinetische Energien bei
dem Verlassen der Metalloberfläche. Die energetischsten Elektronen sind dann die Elektronen, die noch bis kurz
vor der Grenzspannung an die Anode gelangen.
Durch die geringe Verdampfungstemperatur des Kathodenmaterials, wodurch bereits Elektronen herausgelöst werden und sich in der Anode anlagern,
entsteht ein elektrisches Feld.
Deshalb kann ein dem Photostrom entgegengerichteter Strom entstehen.
Es sei angemerkt, dass es sich dabei um deutlich weniger Elektronen handelt, weshalb der Sättigungswert viel schneller erreicht wird.

