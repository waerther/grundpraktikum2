\section{Durchführung}
\label{sec:Durchführung}

\subsection{Vorbereitungsaufgabe}

Als Vorbereitungsaufgabe sollte zunächst die Schallgeschwindigkeit in Luft, destilliertem Wasser und Acryl recherchiert werden \cite{schallgeschw}:
\begin{align*}
    c_\text{Luft} &= 343.2 \unit{\meter} / \unit{\second} \\
    c_\text{Wasser} &= 1480 \unit{\meter} / \unit{\second} \\
    c_\text{Acrylglas} &= 2730 \unit{\meter} / \unit{\second}
\end{align*} 

Weiterhin soll die Wellenlänge und Periode von Acryl errechnet werden. Das erfolgt über die Beziehung $\lambda = c / f$.
Es folgt:
\begin{align*}
    1 \, \unit{\mega\hertz}&: \lambda_1 =  2.73 \unit\meter\\
    2 \, \unit{\mega\hertz}&: \lambda_2 =  1.365\unit\meter\\
    4 \, \unit{\mega\hertz}&: \lambda_3 =  0.6825 \unit\meter
\end{align*}