\section{Diskussion}
\label{sec:Diskussion}

\subsection{Ausmessung des Acrylblocks mittels A-Scan}

Da während der Messung der Störstellen eins und zwei keine weitere Sonden-Frequenz gemacht wurde, 
konnte keine axiale Auflösung zweier benachbarter Störstellen im Acrylblock bestimmt werden.
Des Weiteren soll erwähnt sein, dass die Störstelle 10 beim A- und B-Scan von der Störstelle 11 verdeckt wurde und daher keine Aufnahme von unten möglich war, 
umgekehrt jedoch schon.

Im Vergleich der experimentell bestimmten Durchmesser der Störstellen in \autoref{tab:vgl} ergeben sich die in \autoref{tab:abw} dargestellten Abweichungen.
\begin{table}[H]
    \centering
    \caption{Die berechneten Durchmesser im Vergleich mit den gemessenen.}
    \label{tab:vgl}
    \begin{tabular}{c c c c}
        \toprule
        Stelle &
        $d_\text{Loch} \mathbin{/} \unit{\milli\meter}$ (A-Scan) &
        $d_\text{Loch} \mathbin{/} \unit{\milli\meter}$ (B-Scan) & 
        $d_\text{Loch} \mathbin{/} \unit{\milli\meter}$ \\
        \midrule
                3 &             4,32 &             5,28 &           5,80 \\
                4 &             3,37 &             4,87 &           4,70 \\
                5 &             3,23 &             2,55 &           3,60 \\
                6 &             3,37 &             2,82 &           2,86 \\
                7 &             1,86 &             2,55 &           2,86 \\
                8 &             4,87 &             2,82 &           2,86 \\
                9 &             1,05 &             2,68 &           2,86 \\
                10 &                  &                  &           2,86 \\
                11 &             8,83 &             9,37 &           9,00 \\
        \bottomrule
    \end{tabular}
\end{table}

\begin{table}
    \centering
    \caption{Die relativen Abweichungen der bestimmten Durchmesser.}
    \label{tab:abw}
    \begin{tabular}{c c c}
        \toprule
        Stelle &
        $\increment d_\text{Loch} \mathbin{/} \%$ (A-Scan) &
        $\increment d_\text{Loch} \mathbin{/} \%$ (B-Scan) \\
        \midrule
         3 &  25,52 &                   8,97 \\
         4 &  28,30 &                   3,62 \\
         5 &  10,28 &                  29,17 \\
         6 &  17,83 &                   1,40 \\
         7 &  34,97 &                  10,84 \\
         8 &  70,28 &                   1,40 \\
         9 &  63,29 &                   6,29 \\
         10 &       &                        \\
         11 &  1,89 &                   4,11 \\
        \bottomrule
    \end{tabular} 
\end{table}

Neben der allgemeinen Messung mit der Sonde, fiel das Ablesen der Laufzeiten in \autoref{sec:a-scan} und \autoref{sec:b-scan} schwer,
da der Cursor im Messprogramm verwendet wurde.  
Weitere Fehlerquellen für die sporadischen Abweichungen finden sich in der Herausforderung, klare Aufnahmen beim B-Scan zu machen, 
während die Sonde über den Acrylblock geführt wird.


\subsection{Bestimmung der Lage und Größe der Tumore im Brustmodell}

Im Vergleich zum zweiten Tumor ist bei der Aufnahme des ersten Tumor ein klarer Peak zu erkennen, 
der auf eine härteres Hindernis und daher auf den festen Tumor schließen lässt.
Der zweite hingegen lässt einen weiteren Peak nach dem ersten erkenne, was als ein Hohlraum gedeutet werden kann.
Der Tumor auf der rechten Seite kann daher als Zyste identifiziert werden.
Ein B-Scan, auf dem ein klarer Unterschied der beiden Geschwüre erkenntlich ist, konnte nicht gemacht werden.