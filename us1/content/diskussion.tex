\section{Diskussion}
\label{sec:Diskussion}
Als Ergebnis bei der Schallgeschwindigkeitsermittlung wurde $c_\text{Acryl} = (2740 \pm 270) \unit{\meter / \second}$ ermittelt.
Wird dies mit dem Theoriewert für Acryl von $c = 2730 \unit{\meter / \second}$ verglichen.
Es ergibt sich eine Abweichung von 
\begin{equation*}
    \increment c = \frac{2740  - 2730}{2730} \cdot 100 \simeq 0.37 \%.
\end{equation*}
Weiterhin ergab sich der experimentelle Wert $\lambda = (1.26 \pm 0.12) \cdot 10^{-3} \, \unit\meter$ und $f = 2.174 \, \unit{\mega\hertz}$.
In der Vorbereitung zu dem Versuch wurden die Literaturwerte ermittelt.
Zu dem Theoriewert $\lambda = 1.365 \cdot  10^{-3}$ ergibt sich eine Abweichung von 
\begin{equation*}
    \increment \lambda = \frac{1.365  - 1.26}{1.26} \cdot 100 \simeq 8.33 \%.
\end{equation*}

Die Schallgeschwindigkeitsermittlung bei den verschiedenen Acrylzylindern ergab $c_\text{Acryl} = (2740 \pm 10) \unit{\meter / \second}$.
Auch hier ergibt sich also eine Abweichung von $\increment c \simeq 0.37 \%.$.

Bei dem Durchschallungsverfahren ergab sich eine Schallgeschwindigkeit von $c_\text{Acryl} = (2760 \pm 36)  \, \unit{\meter / \second}$,
woraus die Abweichung 
\begin{equation*}
    \increment c = \frac{2760  - 2730}{2730} \cdot 100 \simeq 1.1 \%
\end{equation*}
folgt.

Für den Dämpfungsfaktor ergab sich $\alpha = 28 \pm 1.3$.
Zusammenfassend kann gesagt werden, dass die Ergebnisse des Experimentes nur geringe Abweichungen zu der Theorie aufweisen und somit als Erfolg angesehen werden kann.