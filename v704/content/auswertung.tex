\section{Auswertung}
\label{sec:Auswertung}

Im folgenden werden wir uns mit der Auswertung der Messdaten beschäftigen.

\subsection[Gamma-Strahlung]{$\gamma$-Strahlung}

Zunächst muss wie in \ref{sec:Durchführung} beschrieben eine Nullmessung durchgeführt werden.
Dabei ergaben sich bei einer Zeit von $t = 900 \unit\second$ eine Impulsrate von $N = 960 \pm 31$,
wobei sich der Fehler nach der Poissonverteilung richtet und somit durch $\increment N = \sqrt{N}$ berechnet wird.
Daraus erfolgt dann eine Aktivitätsrate von $A_0 = 1.07 \pm 0.034 \frac{1}{\unit\second}$, wobei der Fehler nach
der Formel in \autoref{eq:fehlerfortplanzung} berechnet wurde.

Nun soll der Absorbtionskoeffizient von Eisen und Kupfer bestimmt werden.

\subsubsection[Absorbtionskoeffizient von Eisen]{Absorbtionskoeffizient von Eisen}

Die Messdaten zu der Messung von Eisen ist in \autoref{tab:md1} zu finden.