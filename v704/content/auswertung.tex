\section{Auswertung}
\label{sec:Auswertung}

Im folgenden werden wir uns mit der Auswertung der Messdaten beschäftigen.

\subsection[Gamma-Strahlung]{$\gamma$-Strahlung}

Zunächst muss wie in \ref{sec:Durchführung} beschrieben eine Nullmessung durchgeführt werden.
Dabei ergaben sich bei einer Zeit von $t = 900 \unit\second$ eine Impulsrate von $N = 960 \pm 31$,
wobei sich der Fehler nach der Poissonverteilung richtet und somit durch $\increment N = \sqrt{N}$ berechnet wird.
Daraus erfolgt dann eine Aktivitätsrate von $A_0 = 1.07 \pm 0.034 \frac{1}{\unit\second}$, wobei der Fehler nach
der Formel in \autoref{eq:fehlerfortplanzung} berechnet wurde.

Nun soll der Absorbtionskoeffizient von Eisen und Kupfer bestimmt werden.

\subsubsection*{Absorbtionskoeffizient von Eisen}

Die Messdaten zu der Messung von Eisen ist in \autoref{tab:md1blei} zu finden.

\begin{table}
    \center
    \begin{tabular}{c c c}
        \toprule
        $d / \unit{\milli\meter}$ &  $t / \unit\second$ &     n \\ 
        \midrule
            1,2 &  200 & 21610 \\
           10,1 &  200 &  9358 \\
           11,3 &  200 &  8192 \\
           20,0 &  300 &  7304 \\
           21,4 &  300 &  4839 \\
           30,1 &  400 &  2982 \\
           31,3 &  400 &  2726 \\
           21,2 &  400 &  6113 \\
           20,2 &  400 &  6707 \\
           40,2 &  500 &  1959 \\
        \bottomrule
    \end{tabular}
    \caption{Messdaten des Absorbtionskoeffizienten für Eisen}
    \label{tab:md1blei}
\end{table}

subsubsection*{Absorbtionskoeffizient von Kupfer}

\begin{table}
    \center
    \begin{tabular}{c c c}
        \toprule
        $d / \unit{\milli\meter}$ &  $t / \unit\second$ &     n \\
        \midrule
            0,5 &    100 & 11719 \\
            3,0 &    100 & 10162 \\
            3,5 &    100 &  9519 \\
            4,0 &    100 &  9692 \\
            5,0 &    150 & 14304 \\
            5,5 &    150 & 14565 \\
            6,0 &    150 & 14166 \\
           10,0 &    150 & 12274 \\
           13,0 &    200 & 13852 \\
           16,0 &    200 & 12946 \\
        \bottomrule
    \end{tabular}
    \caption{Messdaten des Absorbtionskoeffizienten für Kupfer}
    \label{tab:md2kupfer}
\end{table}

\subsection[Beta-Strahlung]{$\beta$-Strahlung}

\begin{table}
    \center
    \begin{tabular}{c c c}
        \toprule
        $d / \unit{\micro\meter}$ &  $t / \unit\second$ &     n \\
        \midrule
                100 &    200 & 2001 \\
                125 &    200 & 1830 \\
                153 &    200 & 2014 \\
                160 &    200 & 1113 \\
                200 &    400 &  843 \\
                253 &    400 &  325 \\
                302 &    400 &  302 \\
                338 &    500 &  337 \\
                400 &    500 &  339 \\
                444 &    500 &  340 \\
                483 &    500 &  297 \\
        \bottomrule
    \end{tabular}
    \caption{Messdaten des Absorbtionskoeffizienten für Aluminium}
    \label{tab:md3alu}
\end{table}