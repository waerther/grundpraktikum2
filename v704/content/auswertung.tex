\section{Auswertung}
\label{sec:Auswertung}

Im folgenden werden wir uns mit der Auswertung der Messdaten beschäftigen.

\subsection[Gamma-Strahlung]{$\gamma$-Strahlung}

Zunächst muss wie in \ref{sec:Durchführung} beschrieben eine Nullmessung durchgeführt werden.
Dabei ergaben sich bei einer Zeit von $t = 900 \unit\second$ eine Impulsrate von $N = 960 \pm 31$,
wobei sich der Fehler nach der Poissonverteilung richtet und somit durch $\increment N = \sqrt{N}$ berechnet wird.
Daraus erfolgt dann eine Aktivitätsrate von $A_0 = 1.07 \pm 0.034 \frac{1}{\unit\second}$, wobei der Fehler nach
der Formel in \autoref{eq:fehlerfortplanzung} berechnet wurde.

Nun soll der Absorptionskoeffizient von Eisen und Kupfer bestimmt werden.

\subsubsection*{Absorptionskoeffizient von Eisen}

Die Messdaten zu der Messung von Eisen ist in \autoref{tab:md1blei} zu finden.





\begin{table}
    \center
    \begin{tabular}{c c c c}
        \toprule
        $d / \unit{\milli\meter}$ &  $t / \unit\second$ &     n & $A(D) - A_0 \frac{1}{\unit\second}$\\ 
        \midrule
            1,2 &  200 & 21610 $\pm$ 150 &  107.0 $\pm$ 0.7 \\
           10,1 &  200 &  9360 $\pm$ 100 &   45.7 $\pm$ 0.5 \\
           11,3 &  200 &  8190 $\pm$ 90  &   39.9 $\pm$ 0.5 \\
           20,0 &  300 &  7300 $\pm$ 90  & 23.28  $\pm$ 0.28 \\
           21,4 &  300 &  4840 $\pm$ 70  & 15.06  $\pm$ 0.23 \\
           30,1 &  400 &  2980 $\pm$ 60  &  6.39  $\pm$ 0.14 \\
           31,3 &  400 &  2730 $\pm$ 50  &  5.75  $\pm$ 0.13 \\
           21,2 &  400 &  6110 $\pm$ 80  & 14.22  $\pm$ 0.20 \\
           20,2 &  400 &  6710 $\pm$ 80  & 15.70  $\pm$ 0.21 \\
           40,2 &  500 &  1960 $\pm$ 40  &  2.85  $\pm$ 0.09 \\
        \bottomrule
    \end{tabular}
    \caption{Messdaten des Absorptionskoeffizienten für Eisen}
    \label{tab:md1blei}
\end{table}

\subsubsection*{Absorptionskoeffizient von Kupfer}

\begin{table}
    \center
    \begin{tabular}{c c c c}
        \toprule
        $d / \unit{\milli\meter}$ &  $t / \unit\second$ &     n & $A(D) - A_0 \frac{1}{\unit\second}$\\
        \midrule
            0,5 &    100 & 11720 $\pm$ 110 & 116.1 $\pm$ 1.1 \\
            3,0 &    100 & 10160 $\pm$ 100 & 100.6 $\pm$ 1.0 \\
            3,5 &    100 &  9520 $\pm$ 100 &  94.1 $\pm$ 1.0 \\
            4,0 &    100 &  9690 $\pm$ 100 &  95.9 $\pm$ 1.0 \\
            5,0 &    150 & 14300 $\pm$ 120 &  94.3 $\pm$ 0.8 \\
            5,5 &    150 & 14560 $\pm$ 120 &  96.0 $\pm$ 0.8 \\
            6,0 &    150 & 14170 $\pm$ 120 &  93.4 $\pm$ 0.8 \\
           10,0 &    150 & 12270 $\pm$ 110 &  80.8 $\pm$ 0.7 \\
           13,0 &    200 & 13850 $\pm$ 120 &  68.2 $\pm$ 0.6 \\
           16,0 &    200 & 12950 $\pm$ 110 &  63.7 $\pm$ 0.6 \\
        \bottomrule
    \end{tabular}
    \caption{Messdaten des Absorptionskoeffizienten für Kupfer}
    \label{tab:md2kupfer}
\end{table}

\subsection[Beta-Strahlung]{$\beta$-Strahlung}

\begin{table}
    \center
    \begin{tabular}{c c c c}
        \toprule
        $d / \unit{\micro\meter}$ &  $t / \unit\second$ &     n & $A(D) - A_0 \frac{1}{\unit\second}$\\
        \midrule
                100 &    200 &   2000  $\pm$  40 &   9.42 $\pm$ 0.23 \\
                125 &    200 &   1830  $\pm$  40 &   8.57 $\pm$ 0.21 \\
                153 &    200 &   2010  $\pm$  40 &   9.49 $\pm$ 0.23 \\
                160 &    200 & 1113 $\pm$ 33 &   4.98 $\pm$ 0.17 \\
                200 &    400 &  843 $\pm$ 29 &   1.52 $\pm$ 0.07 \\
                253 &    400 &  325 $\pm$ 18 &   0.23 $\pm$ 0.04 \\
                302 &    400 &  302 $\pm$ 17 &   0.17 $\pm$ 0.04 \\
                338 &    500 &  337 $\pm$ 18 &   0.09 $\pm$ 0.04 \\
                400 &    500 &  339 $\pm$ 18 &   0.09 $\pm$ 0.04 \\
                444 &    500 &  340 $\pm$ 18 &   0.10 $\pm$ 0.04 \\
                482 &    500 &  297 $\pm$ 17 &   0.011$\pm$ 0.034\\
        \bottomrule
    \end{tabular}
    \caption{Messdaten des Absorptionskoeffizienten für Aluminium}
    \label{tab:md3alu}
\end{table}