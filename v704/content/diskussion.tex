\section{Diskussion}
\label{sec:Diskussion}

Als experimentelles Ergebnis ergab sich für die Absorptionskoeffizienten
\begin{align*}
    \mu_\text{Blei} &= 95.25 \pm 2.74 \frac{1}{\unit\meter}&\text{und}&&\mu_\text{Kupfer} &= 37.32 \pm 2.74 \frac{1}{\unit\meter}
\end{align*}
und als Theoriewerte 
\begin{align*}
    \mu_\text{Blei, T} &= 69.34 \frac{1}{\unit\meter}&\text{und}&&\mu_\text{Kupfer, T} &= 63.1 \frac{1}{\unit\meter}.
\end{align*} 
Daraus ergibt sich jeweils eine Abweichung von
\begin{align*}
    \increment \mu_\text{Blei} &= 37.37 \%&\increment \mu_\text{Kupfer} &= 39.12 \%.
\end{align*}
% Das ist nicht als Erfolg zu betrachten, da in diesem Experiment eigentlich keine Messfehler seitens des
% Experimentators entstehen können, da die Messwerte nur abgelesen werden müssen und bis auf die Messung durch den Nonius
% alles vorgegeben ist.

Als Fehlerquelle kann die geringe Variation der Dicken angegeben werden, oder allgemein die geringe Anzahl an Durchführungen.
Da die Fehler jedoch, wie in \autoref{fig:plot1} und \autoref{fig:plot2} zu erkennen, relativ gering sind, ist anzunehmen dass hier nicht nur Compton-Streuung vorliegt
und andere Effekte beitragen.

% Ein größerer Messzeitraum würde die Messung ebenfalls verbessern, da die Zerfälle statistisch gesehen besser verteilt sein würden und somit bessere Ergebnisse liefern würden.\\

Bei dem $\beta$-Strahler wird als Vergleichswert $E_\text{Theorie} = 0.293 \unit{\mega\eV}$ genommen \cite{technetium99}.
Als experimentellen Wert ergab sich $E_\text{max} = (0.62 \pm 0.2) \unit{\mega\eV}$.
Daraus ergibt sich eine Abweichung von
\begin{equation*}
    \increment E = 111.6 \%.
\end{equation*}
Als Fehlerquelle kann auch hier die geringe Anzahl an Messwerten aufgezählt werden, wodurch sich die Präzession der Regressionsgeraden verbessern würde.
Dabei wären vor allem Messungen im Bereich $(0 - 200) \unit{\micro\meter}$ nötig, da die Hintergrundstrahlung ziemlich gut gemessen wurde.
Das lässt sich an der sehr geringen Steigung des ersten Fits erkennen.  