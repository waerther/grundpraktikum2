\section{Diskussion}
\label{sec:Diskussion}

Als experimentelles Ergebnis ergab sich für die Absorbtionskoeffizienten
\begin{align*}
    \mu_\text{Blei} &= 95.25 \pm 2.74 \frac{1}{\unit\meter}&\text{und}&&\mu_\text{Kupfer} &= 37.32 \pm 2.74 \frac{1}{\unit\meter}
\end{align*}
und als Theoriewerte 
\begin{align*}
    \mu_\text{Blei, T} &= 69.34 \frac{1}{\unit\meter}&\text{und}&&\mu_\text{Kupfer, T} &= 63.1 \frac{1}{\unit\meter}.
\end{align*} 
Daraus ergibt sich jeweils eine Abweichung von
\begin{align*}
    \increment \mu_\text{Blei} &= 37.37 \%&\increment \mu_\text{Blei} &= 39.12 \%.
\end{align*}
Das ist nicht als Erfolg zu betrachten, da in diesem Experiment eigentlich keine Messfehler seitens des
Experimentators entstehen können, da die Messwerte nur abgelesen werden müssen und bis auf die Messung durch den Nonius
alles vorgegeben ist.
Als Fehlerquelle kann die geringe Variation der Dicken angegeben werden, oder allgemein die geringe Anzahl an Durchführungen.
Ein größerer Messzeitraum würde die Messung ebenfalls verbessern, da die Zerfälle statistisch gesehen besser verteilt sein würden und somit bessere Ergebnisse liefern würden.\\

Bei dem $\beta$-Strahler wird als Vergleichswert \cite{technetium99} $E_\text{Theorie} = 0.293 \unit{\mega\eV}$ genommen.
Als experimentellen Wert ergab sich $E_\text{max} = (0.97+/-0.32) \unit{\mega\eV}$.
Daraus ergibt sich eine Abweichung von
\begin{equation*}
    \increment E = 231.05 \%.
\end{equation*}
Offensichtlich viel das Ergebnis hier noch weitaus schlechter aus.
Anzumerken ist, dass bei der Aufstellung des Plots in \autoref{fig:plot3} der erste Messwert ausgelassen wurde, da sonst
$R_\text{max}$ sehr hohe Werte angenommen hätte.
Vermutlich hätten auch mehrere Messwerte der Hintergrundstrahlung dazu beigetragen, eine aussagekräftigere Regressionsgerade zu bekommen.
Auch hätten bei verschiedenen Dicken die Messzeit erhöht werden können und mehrere Messreihen aufgestellt werden sollen.