\section{Diskussion}
\label{sec:Diskussion}

\subsection{Bestimmung der Wellenlänge des Lasers}

Die berechnete Wellenlänge von $\overline{\lambda} = \qty{3404.29(750.59)}{\nano\meter}$
weicht $\qty{436.11}{\percent}$ von der tatsächlichen Wellenlänge des Laser $\lambda = \qty{635}{\nano\meter}$.
Diese enorme Abweichung ist in erster Linie auf einen systematischen Fehler zurückzuführen.
Besonders bei dem Vergleich mit anderen Versuchsiterationen und -aufbauten fällt auf, 
dass die gemessenen Interferenzmaxima weit unter dem erwarteten Intervall von 2500 bis 3500 liegen.
In zweiter Linie können Störquellen wie äußere Lichteinflüsse auf die Messzelle oder Vibrationen am Gerät
zudem vorerst nicht erfasst und daher ausgeschlossen werden,
da diese die Anzahl der Interferenzmaxima wenn überhaupt erhöhen würden.


\subsection{Berechnung des Brechungsindex von Luft}

Im Vergleich mit dem Literaturwert des Brechungsindex von Luft $n_\text{lit} = 	1,000292$ nach \cite{brechungsindex}
ergibt sich eine Abweichung von $\qty{0.0375}{\percent}$. 
Die geringe Abweichung dieser Messung ist bezüglich des Fehler aus voriger Messung überraschend.
Da in diesem Versuchsteil keine Verschiebung des Spiegels stattfand, sonder nur der Druck variiert wurde,
bewegt sich der vermutete systematische Fehler von der Messzelle bzw. dem Zählwerk weg auf die Mikrometerschraube hin.
Obwohl die berechnete Wellenlänge eine große Abweichung aufweist, dominiert in \autoref{eq:index} der Druckterm
und führt daher zu einer besseren Berechnung. 
Bei dem ersten gemessen Wert in \autoref{tab:indizes} fällt ebenfalls die hohe Anzahl an Interferenzmaxima auf,
welche auf Vibrationen des Gerätes bei der Druckänderung zurückzuführen sind.