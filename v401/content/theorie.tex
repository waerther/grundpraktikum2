\section{Theorie}
\label{sec:Theorie}

Zunächst werden die theoretischen Grundlagen des Versuchs erklärt.
Dafür müssen die Begriffe der Interferenz und Kohärenz erklärt werden.

\subsection{Interferenz von Lichtwellen}
Wie aus der Physik über Elektrodynamik bekannt ist, ist Licht eine elektromagnetische Welle.
Diese Welle lässt sich über die Maxwellschen Gleichungen genau beschreiben.
Die elektrische Feldstärke eine solchen elektromagnetischen Welle ist genau über die Gleichung
\begin{equation}
    \vec{E} = \vec{E_0} \text{cos}\left( kx - \omega t - \delta \right)
\end{equation} 
charakterisiert. Dabei ist $k = \frac{2 \pi}{\lambda}$ die Wellenzahl, $\omega$ die Kreisfrequenz und $\omega$ der Phasenwinkel.
Die Intensität einer solchen elektromagnetischen Welle wird nach
\begin{equation}
    I = \text{const} |\vec{E}|^2
\end{equation}
berechnet. Überlagern sich zwei Wellen ergibt sich daraus für die Intensität
\begin{equation}
    I_\text{ges} = 2 \text{const} \vec{E}^2_0 \cdot \left( 1 + \text{cos} \left( \delta_2 - \delta_1 \right) \right).
\end{equation}
Anzumerken ist dabei, dass bei 
\begin{equation*}
    \delta_2 - \delta_1 = (2n + 1) \pi, n \in \mathbb{N}
\end{equation*}
die Intensität null ist.
\subsection{Kohärentes Licht}
Da diese Lichtquellen im Normalfall nicht von Lasern emittiert werden, sind die
$\delta_2$ und $\delta_1$ in der Regel statistisch verteilt über die Zeit.
Deshalb ist verschwindet der Interferenzterm und es lässt sich so keine Interferenz beobachten.
Allerdings stehen eben solche Laser (light amplification by stimulated emission of radiation) heute zu light amplification by stimulated emission of radiation) Verfügung.
So lässt sich \textbf{kohärentes}, interferenzfähiges Licht herstellen.
Dieses wird durch Atome im Gleichtakt emittiert.
Kohärentes Licht besitzt also nach diesen Eigenschaften ein festes $k, \omega$ und $\delta$.