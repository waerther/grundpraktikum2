\section{Auswertung}
\label{sec:Auswertung}

\subsection{Bestimmung der Wellenlänge des Lasers}

Die Bestimmung der Wellenlänge nach \autoref{eq:delta_d}, 
wobei die variierte Spiegelverschiebung bei $\increment d = \qty{5}{\milli\meter}$ beträgt.
Zusätzlich wird neben den Interferenzmaxima $z$ die Hebelübersetzung als Faktor $\frac{1}{Ü}$ berücksichtigt,
da die abgelesene Verschiebung des Spiegels übersetzungsbedingt von der tatsächlichen verschieden ist.
Die Hebelübersetzung beträgt hierbei 
\begin{equation*}
    Ü = 5,046 \, .
\end{equation*}

Die sich somit ergebende Gleichung für die Wellenlänge ergibt sich zu
\begin{equation} \label{eq:lambda}
    \lambda = \frac{2 \cdot \increment d}{z} \, .
\end{equation}

Die Messdaten sowie die berechneten Wellenlängen sind in \autoref{tab:lambda} dargestellt.
\begin{table}
    \centering
    \caption{Messdaten zur Berechnung der Wellenlänge des Lasers.}
    \label{tab:lambda}
    \begin{tabular}{c c}
        \toprule
        z &  $\lambda \mathbin{/} \mathrm{nm}$ \\
        \midrule
        600 & 3302,95 \\
        556 & 3564,33 \\
        430 & 4608,76 \\
        420 & 4718,49 \\
        560 & 3538,87 \\
        640 & 3096,51 \\
        866 & 2288,42 \\
        677 & 2927,28 \\
        566 & 3501,36 \\
        794 & 2495,93 \\
        \bottomrule
    \end{tabular}
\end{table}

Aus einer Mittelung ergibt sich die Wellenlänge des Lasers zu
\begin{equation*}
    \overline{\lambda} = \qty{3404.29(750.59)}{\nano\meter} \, .
\end{equation*}


\subsection{Berechnung des Brechungsindex von Luft}

Zur Bestimmung des Brechungsindex von Luft werden zunächst die Normalbedingungen, 
die Größe der Messzelle und die Umgebungstemperatur benötigt:
\begin{align*}
    \text{Normaldruck}          : \quad p_{0}   &= \qty{1,0132}{bar} \\
    \text{Normaltemperatur}     : \quad T_{0}   &= \qty{273,15}{\kelvin} \\
    \text{Umgebungstemperatur}  : \quad T       &= \qty{293.15}{\milli\ampere} \\
    \text{Größe der Messzelle}  : \quad b       &= \qty{50}{\milli\meter} 
\end{align*}

Die Berechnung nach \autoref{eq:index} unter diesen Normalbedingungen bei einem angenommenen Kammerdruck von $p = p_{0}$
und einem erniedrigtem Druck $p' = \qty{600}{mbar}$ erfolgt schließlich durch
\begin{equation}
    n\left(p_{0}, T_{0}\right)=1+\frac{z \overline{\lambda}}{2 b} \frac{T}{T_{0}} \frac{p_{0}}{p_{0}-p^{\prime}} \, .
\end{equation}

Die gemessenen Interferenzmaxima und die berechneten Berechnungsindizes sind in \autoref{tab:indizes} dargestellt.
\begin{table}
    \centering
    \caption{Messdaten zur Berechnung des Brechungsindex von Luft.}
    \label{tab:indizes}
    \begin{tabular}{c c}
        \toprule
           z &        n \\
        \midrule
        80 & 1,071670 \\
        30 & 1,026876 \\
        50 & 1,044794 \\
        34 & 1,030460 \\
        34 & 1,030460 \\
        41 & 1,036731 \\
        34 & 1,030460 \\
        35 & 1,031356 \\
        \bottomrule
    \end{tabular} 
\end{table}

Eine Mittelung dieser Werte ergibt einen Brechungsindex von
\begin{equation*}
    \overline{n} = \qty{1.038(0.014)} \, .
\end{equation*}