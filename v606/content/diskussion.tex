\section{Diskussion}
\label{sec:Diskussion}

Zum Vergleich der theoretisch und experimentell bestimmten Werten, 
wird in \autoref{tab:vergleich} die relative Abweichung der beiden dargestellt.
\begin{table}
    \centering
    \caption{Die relative Abweichungen von der theoretischen Suszeptibilität.}
    \label{tab:vergleich}
    \begin{tabular}{c c c}
        \toprule
        Stoff &  
        $\increment \chi_\text{U} \mathbin{/} \%$ & 
        $\increment \chi_\text{R} \mathbin{/} \%$ \\
        \midrule
        $\ce{Dy2O3}$ &   476,74 &    14,87 \\
        $\ce{Gd2O3}$ &   431,94 &    22,50 \\
        $\ce{Nd2O3}$ &    58,82 &    40,66 \\
        \bottomrule
    \end{tabular}
\end{table}

Bei der Messung der Filterfrequenz erschwerte zum einen der Bandpass-Filter das Schaffen eines klaren Sinus-Signals, 
was nur durch wiederholtes An- und Ausschalten gelöst werden konnte.
Zum Anderen schwankte die Messanzeige des Selektivverstärkers stark, 
weswegen die Annäherung der Filterfrequenz und der maximalen Spannung äußerst ungenau ist.

Da die Suszeptibilität stark von der Temperatur des Stoffes abhängt, kann das Einlegen der Proben als Fehlerquelle betrachtet werden.
Die vermutlich jedoch größte Fehlerquelle stellt der Abgleich der Brückenspannung dar, 
da diese nur minimiert werden und nicht auf Null reduziert werden konnte, trotz des Selektivverstärkers. 
Dies hat auch als Ursache, dass sich eine genauere Messung mithilfe der Widerstände erzielen lässt.