\section{Diskussion}
\label{sec:Diskussion}

\subsection{Die Kennlinien}
Die aufgenommenen Kennlinien in \autoref{fig:k} zeigen bei höherem Heizstrom einen Anstieg des gemessenen Stromes.
Dies ist mit dem zuvor erklärten glühelektrischen Effekt vereinbar, wonach der bei höheren Heizströmen mehr Elektronen austreten.
Im Vergleich mit der in \autoref{fig:KennlinieTheorie} dargestellten Theoriekurve zeigt sich eine große Ähnlichkeit mit den Kennlinien.
Während bei den Kennlinien in den \hyperref[fig:k]{Kennlinien \ref{fig:k1} - \ref{fig:k2}} sich ein deutlicher Sättigungstrom einstellt,
lässt sich in den \hyperref[fig:k]{Kennlinien \ref{fig:k3} - \ref{fig:k5}} höchstens annähernd eine Asymptote ablesen.
Bei ausreichend vielen Messdaten stelle sich mithilfe einer Extrapolation und einem dadurch bestimmten Wendepunkt eine bessere Möglichkeit dar,
den Sättigungsstrom zu finden.

\subsection{Überprüfung des Raumladungsgesetzes}
Um das Raumladungsgesetz aus \autoref{eq:lsr} zu kontrollieren, wird der Exponent abgeschätzt und mit der Theorie verglichen.
Nach der Messung ergibt sich der Exponent zu $b = \num{1.232(0.031)} \,$, 
was eine Abweichung von $\qty{18}{\percent}$ vom Wert $b = 1,5$ darstellt.
Die hohe Abweichung schließt auf die Empfindlichkeit des Gerätes beim Einstellen, denn bereits
geringe Veränderungen am Spannungsrad oder Bewegungen außerhalb der Hochvakuum-Diode tragen merklich zur Messgenauigkeit bei.

\subsection{Anlaufstromgebiet}
Die berechnete Temperatur $T = \qty{2131(70)}{\kelvin}$ liegt innerhalb des angegebenen Intervalls von $1000 - 3000 \, \mathrm{K}$. 
Der Verlauf des Anlaufstroms in \autoref{fig:anlaufstrom} ist wie zu erwarten exponentiell und
hat Ähnlichkeit mit dem Raumladungsgebiet des theoretischen Verlaufs in \autoref{fig:KennlinieTheorie}.
Dabei ist anzumerken, dass die gemessene Kurve fällt, weil die Gegenspannung positiv aufgetragen wurde.

\subsection{Kathodentemperatur}
Wie auch die Stromverläufe bei den Kennlinien, steigt die Temperatur der Kathode mit höherem Heizstrom an und liegt im erwarteten Intervall.
Fehlerquellen stellen dabei abgeschätzte Größen wie Kathodenoberfläche, 
deren Emissionsgrad sowie die Wärmeleitung der Fadenhalterung im Bereich $N_{\text{WL}} = 0,9$ bis $1,0 \, \mathrm{W}$.

\subsection{Austrittsarbeit von Wolfram}
Die ermittelte Austrittsarbeit von Wolfram beträgt $\overline{W_{\text{A}}} = \qty{3.12(4)}{\eV} \,$, 
dieser weicht $\qty{31}{\percent}$ vom Literaturwert $W_{\text{A, Lit}} = \qty{4.55}{\eV} \,$ \cite[14]{austrittsarbeit} ab.
Erneut liegen die Fehlerquellen in dem Abschätzen der zuvor genannten Größen und den daraus fehlerbehafteten Temperaturen.