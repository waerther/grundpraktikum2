\section{Diskussion}
\label{sec:Diskussion}

\subsection{Die Kennlinien}
Der gemessene Strom steigt wie in \autoref{fig:k} zu erkennen immer weiter an, je höher der Heizstrom gewählt wird.
Dies ist zu erwarten, denn aufgrund des zuvor erklärten gLühelektrischen Effektes treten bei höheren Heizströmen auch mehr Elektronen aus.
Die Kurven sind auch der in \autoref{fig:KennlinieTheorie} dargestellten Theoriekurve einer solchen Messung ähnlich.
Es stellt sich bei den \hyperref[fig:k]{Kennlinien \ref{fig:k1} - \ref{fig:k3}} ein deutlicher Sättigungsstrom ein.
Bei den anderen beiden Messungen, dargestellt in \hyperref[fig:k]{Kennlinien \ref{fig:k4} - \ref{fig:k5}}, lässt sich jedoch maximal eine Tendenz erkennen. 
Dies ist ebenfalls angedeutet durch die abflachende Extrapolation und den abgeschätzten Wendepunkt.

\subsection{Überprüfung des Raumladungsgesetzes}
Um das Raumladungsgesetz aus \autoref{eq:Langmuir} zu kontrollieren, wird der Exponent abgeschätzt und mit der Theorie verglichen.
Nach der Messung ergibt sich ein Wert von $1.163\pm-0.034$.
Das ergibt eine Abweichung von $22.47 \%$.
Die hohe Abweichung ergibt sich durch die Empfindlichkeit des Gerätes beim Einstellen, denn bereits
geringe Veränderungen am Rädchen trugen bereits zu verhältnismäßig großen Änderungen bei.

\subsection{Anlaufstromgebiet}
Die berechnete Temperatur $T = \qty{2131(70)}{\kelvin}$ liegt innerhalb des angegebenen Intervalls
von $1000 - 3000 \, \mathrm{K}$. Der Verlauf des Anlaufstroms in \autoref{fig:anlaufstrom} ist wie zu erwarten exponentiell
und ähnelt sehr stark dem theretischen Verlauf in \autoref{fig:KennlinieTheorie}.
Der einzige Unterschied ist, dass die Kurve in der Theorie steigt, aber die gemessene abfällt.
Dies liegt daran, dass die Gegenspannung positiv aufgetragen wurde.