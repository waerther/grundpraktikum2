\section{Auswertung}
\label{sec:Auswertung}

Im folgenden werden die Messungen, dargestellt in \autoref{fig:kurve1} bis \autoref{fig:kurve4}, ausgewertet.
Dafür wird zunächst die mittlere freie Weglänge bestimmt.
Dann wird die Energieverteilung der beschleunigten Elektronen errechnet.
% Franck-hertz_Kurve?


\subsection{Die mittlere freie Weglänge}
Mit \autoref{eq:tbd} und \autoref{eq:tbd} wird die mittlere freie Weglänge der beschleunigten Elektronen,
die aus der Heizkathode herausgelöst werden, berechnet.
Ebenfalls wird das Verhältnis $a / \bar{\omega}$ bestimmt.
Die Ergebnisse der Rechnung sind in \autoref{tab:tabelle1} dargestellt.

\begin{table} [H]
  \centering
  \caption{Die mittlere freie Weglänge.}
  \label{tab:tabelle1}
  \begin{tabular}{c c c}
      \toprule
      Temperatur $T / \unit\kelvin$ & $\bar{\omega} / \unit\meter$ & $\frac{a}{\bar{\omega}}$ \\
      \midrule 
      298,35 & 0.538412049 & 0.01857314\\
      417,95 & 0.000736088 & 13.5853149\\
      433,25 & 0.000411729 & 24.2877893\\
      448,55 & 0.000239611 & 41.7343021\\
      \bottomrule
  \end{tabular}
\end{table}

\subsection{Differentielle Energieverteilung}
Nun werden die Messdaten der ersten beiden Messungen ausgewertet.
Dabei handelt es sich um \autoref{fig:kurve1} und \autoref{fig:kurve2}, wobei die Graphen von einem X-Y-Schreiber erstellt wurden.
