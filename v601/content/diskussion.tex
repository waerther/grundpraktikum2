\section{Diskussion}
\label{sec:Diskussion}

Zunächst wurde die differentielle Energieverteilung der Elektronen ermittelt.
Dabei wurde festgestellt, dass die meisten Elektronen bei einer konstanten Beschleunigungsspannung von $11 \unit\V$ und bei Raumtemperatur eine Energie von $E = 7,75 \unit\eV$ besitzen.
Daraus wurde dann auch das Kontaktpotential zu $3,25 \unit\V$ ermittelt.
Bei einer Temperatur von $T = 433,25 \unit\K$ ergab sich für $U_\text{B, eff} = 1 \unit\V$.

Die Franck-Hertz-Kurven ergaben für die Energie durch Abrundung aufgrund der Fehler für beide Messungen den selben Wert,
nämlich
\begin{equation}
    E = (5,5 \pm 0.8)\unit\eV \, .
\end{equation}
Als Vergleich wird ein Theoriewert von $4,9 \unit\eV$ angenommen \cite{Franck_Hertz}.
Daraus ergibt sich dann eine Abweichung von 
\begin{equation}
    \increment E = \frac{5,5 - 4,9}{4,9} \cdot 100 = 12,24 \%.
\end{equation}

Als Fehlerquellen können in diesem Experiment mehrere Quellen angegeben werden.
Zunächst einmal konnte die Temperatur nie exakt gehalten werden, sondern schwankte oft in einem Bereich von einigen Grad.
Dadurch änderte sich diese natürlich auch während der Messung.
Außerdem kann die Schwierigkeit der Findung der passenden Einstellung des X-Y-Schreibers als Fehlerquelle angesehen werden, da die Präzession der Messung stark darauf beruht.