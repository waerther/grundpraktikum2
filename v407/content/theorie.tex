\section{Theorie}
\label{sec:Theorie}

Für die Herleitung die Winkelabhängkeit der Intensität werden zunächst die Maxwellgleichngen
\begin{align}
    \nabla \times \vec{H}=\vec{j}+\varepsilon \varepsilon_{0} \frac{ \dif{\vec{E}}}{ \dif{t}} 
    \quad & \text{und} \quad 
    \nabla \times \vec{E}=-\mu \mu_{0} \frac{\dif{\vec{H}}}{\dif{t}} 
\end{align}
betrachtet. Unter Verwendung der Ausdrücke für die elektrische und magnetische Arbeit
\begin{align}
    W_{\text{el}}   :=  \frac{1}{2} \varepsilon \varepsilon_{0} \vec{E}^{2} 
    \quad & \text{und} \quad 
    W_{\text{mag}}  :=  \frac{1}{2} \mu_{0} \vec{H}^{2}
\end{align}
beziehungsweise die Energie der Strahlungsleistung pro Volumen. 
Daraus ergibt sich ein Ausdruck für den \textit{Poynting Vektor.} 
