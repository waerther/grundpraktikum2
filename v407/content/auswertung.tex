\section{Auswertung}
\label{sec:Auswertung}

Nun sollen die Brechungsindizes $n$ bestimmt werden.
Diese werden bei paralleler und senkrechter Polarisation benötigt.
Die beiden Gleichungen werden aus \autoref{eq:tbd} und \autoref{eq:tbd} extrahiert.


\subsection{Der Brechungsindex bei Polarisation senkrechter Ausrichtung}

Allgemein gilt die Proportionalität 
\begin{equation}
  \frac{I_r}{I_e} \thicksim \frac{E_r^2}{E_e} = E^2.
\end{equation}
Durch diese Beziehung lässt sich \autoref{eq:tbd} zu der Form
\begin{equation*}
  E_{\perp}=\left|\frac{\left(\cos ^{2}(\alpha) - \sqrt{n^{2}-\sin ^{2}(\alpha)} \right)^{2}}{n^{2}-1}\right|
\end{equation*}
umstellen.
Diese lässt sich nach $n$ umformen.
Das ergibt
\begin{equation}
  \sqrt{1 + \frac{4 E \cos ^2(\alpha)}{(E - 1)^2}}.
\end{equation}
In \autoref{tab:mess1} sind die Messwerte und das jeweilige errechnete $n$ aufgetragen.

Wird $n$ gemittelt ergibt sich ein Brechungsindex von 
\begin{equation*}
  n_{\perp} = 4.8 ± 8.9. 
\end{equation*}


\begin{table}
  \centering
  \caption{Die Messwerte und die Ergebnisse der Rechnung.}
  \label{tab:mess1}
  \begin{tabular}{c c c c c c c c}
    \toprule
    $\alpha / °$ &     $I / \unit{\milli\ampere}$ & $\frac{I_\text{gem}}{I_\text{ref}}$ & n &$\alpha / °$ &    $I / \unit{\milli\ampere}$ & $\frac{I_\text{gem}}{I_\text{ref}}$& n \\
    \midrule
       6,0 & 0,013 &  0,092857 & 1,167118 &48 & 0,086 & 0,614286 &  2,218190 \\
       8,0 & 0,015 &  0,107143 & 1,193855 &50 & 0,088 & 0,628571 &  2,211720 \\
      10,0 & 0,015 &  0,107143 & 1,191895 &52 & 0,093 & 0,664286 &  2,302198 \\
      12,0 & 0,020 &  0,142857 & 1,261561 &54 & 0,096 & 0,685714 &  2,322735 \\
      14,0 & 0,021 &  0,150000 & 1,272628 &56 & 0,100 & 0,714286 &  2,377642 \\
      16,0 & 0,025 &  0,178571 & 1,328640 &58 & 0,106 & 0,757143 &  2,518670 \\
      18,0 & 0,027 &  0,192857 & 1,353637 &60 & 0,106 & 0,757143 &  2,399442 \\
      20,0 & 0,030 &  0,214286 & 1,393850 &62 & 0,113 & 0,807143 &  2,589478 \\
      22,0 & 0,033 &  0,235714 & 1,433663 &64 & 0,116 & 0,828571 &  2,593267 \\
      24,0 & 0,038 &  0,271429 & 1,507659 &66 & 0,120 & 0,857143 &  2,646371 \\
      26,0 & 0,040 &  0,285714 & 1,529146 &68 & 0,126 & 0,900000 &  2,833082 \\
      28,0 & 0,047 &  0,335714 & 1,643088 &70 & 0,130 & 0,928571 &  2,901770 \\
      30,0 & 0,050 &  0,357143 & 1,682241 &72 & 0,130 & 0,928571 &  2,656563 \\
      32,0 & 0,058 &  0,414286 & 1,829007 &74 & 0,140 & 1,000000 &  3,278201 \\
      34,0 & 0,060 &  0,428571 & 1,845555 &76 & 0,142 & 1,014286 &  3,143644 \\
      36,0 & 0,066 &  0,471429 & 1,956716 &78 & 0,146 & 1,042857 &  3,249584 \\
      38,0 & 0,068 &  0,485714 & 1,969244 &80 & 0,160 & 1,142857 &  8,171412 \\
      40,0 & 0,070 &  0,500000 & 1,978976 &82 & 0,166 & 1,185714 & 46,355221 \\
      42,0 & 0,080 &  0,571429 & 2,213412 &84 & 0,166 & 1,185714 & 34,822183 \\
      44,0 & 0,080 &  0,571429 & 2,157172 &86 & 0,166 & 1,185714 & 23,250316 \\
      46,0 & 0,083 &  0,592857 & 2,189004 &88 & 0,166 & 1,185714 & 11,664424 \\
    \bottomrule
    \end{tabular}
\end{table}

\subsection{Der Brechungsindex bei Polarisation paralleler Ausrichtung}

