\section{Auswertung}
\label{sec:Auswertung}

Nun sollen die Brechungsindizes $n$ bestimmt werden.
Diese werden bei paralleler und senkrechter Polarisation benötigt.
Die beiden Gleichungen werden aus \autoref{eq:tbd} und \autoref{eq:tbd} extrahiert.


\subsection{Der Brechungsindex bei Polarisation senkrechter Ausrichtung}

Allgemein gilt die Proportionalität 
\begin{equation}
  \frac{I_r}{I_e} \thicksim \frac{E_r^2}{E_e} = E^2.
\end{equation}
Durch diese Beziehung lässt sich \autoref{eq:tbd} zu der Form
\begin{equation*}
  E_{\perp}=\left|\frac{\left(\cos ^{2}(\alpha) - \sqrt{n^{2}-\sin ^{2}(\alpha)} \right)^{2}}{n^{2}-1}\right|
\end{equation*}
umstellen.
Diese lässt sich nach $n$ umformen.
Das ergibt
\begin{equation}
  \sqrt{1 + \frac{4 E \cos ^2(\alpha)}{(E - 1)^2}}.
\end{equation}
In \autoref{tab:tbd} sind die Messwerte und das jeweilige errechnete $n$ aufgetragen.

Wird $n$ gemittelt ergibt sich ein Brechungsindex von 
\begin{equation*}
  n_{\perp} = tbd
\end{equation*}

\subsection{Der Brechungsindex bei Polarisation paralleler Ausrichtung}