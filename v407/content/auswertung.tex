\section{Auswertung}
\label{sec:Auswertung}

Nun sollen die Brechungsindizes $n$ bestimmt werden.
Diese werden bei paralleler und senkrechter Polarisation benötigt.
Die beiden Gleichungen werden aus \autoref{eq:tbd} und \autoref{eq:tbd} extrahiert.


\subsection{Der Brechungsindex bei Polarisation senkrechter Ausrichtung}

Allgemein gilt die Proportionalität 
\begin{equation}
  \frac{I_r}{I_e} \thicksim \frac{E_r^2}{E_e} = E^2.
\end{equation}
Durch diese Beziehung lässt sich \autoref{eq:tbd} zu der Form
\begin{equation*}
  E_{\perp}=\left|\frac{\left(\cos ^{2}(\alpha) - \sqrt{n^{2}-\sin ^{2}(\alpha)} \right)^{2}}{n^{2}-1}\right|
\end{equation*}
umstellen.
Diese lässt sich nach $n$ umformen.
Das ergibt
\begin{equation}
  \sqrt{1 + \frac{4 E \cos ^2(\alpha)}{(E - 1)^2}}.
\end{equation}
In \autoref{tab:mess1} sind die Messwerte und das jeweilige errechnete $n$ aufgetragen.

Wird $n$ gemittelt ergibt sich ein Brechungsindex von 
\begin{equation*}
  n_{\perp} = tbd
\end{equation*}



\begin{table}
  \centering
  \caption{Die Messwerte und die Ergebnisse der Rechnung.}
  \label{tab:mess1}
  \begin{tabular}{c c c c}
    \toprule
    $\alpha / °$ &     $I / \unit{\milli\ampere}$ &  $\alpha / °$ &    $I / \unit{\milli\ampere}$ \\
    \midrule
       6,0 & 0,013 &  48 & 0,086 \\   
       8,0 & 0,015 &  50 & 0,088 \\
      10,0 & 0,015 &  52 & 0,093 \\
      12,0 & 0,020 &  54 & 0,096 \\
      14,0 & 0,021 &  56 & 0,100 \\
      16,0 & 0,025 &  58 & 0,106 \\
      18,0 & 0,027 &  60 & 0,106 \\
      20,0 & 0,030 &  62 & 0,113 \\
      22,0 & 0,033 &  64 & 0,116 \\
      24,0 & 0,038 &  66 & 0,120 \\
      26,0 & 0,040 &  68 & 0,126 \\
      28,0 & 0,047 &  70 & 0,130 \\
      30,0 & 0,050 &  72 & 0,130 \\
      32,0 & 0,058 &  74 & 0,140 \\
      34,0 & 0,060 &  76 & 0,142 \\
      36,0 & 0,066 &  78 & 0,146 \\
      38,0 & 0,068 &  80 & 0,160 \\
      40,0 & 0,070 &  82 & 0,166 \\
      42,0 & 0,080 &  84 & 0,166 \\
      44,0 & 0,080 &  86 & 0,166 \\
      46,0 & 0,083 &  88 & 0,166 \\    
    \bottomrule
    \end{tabular}
\end{table}


\subsection{Der Brechungsindex bei Polarisation paralleler Ausrichtung}

