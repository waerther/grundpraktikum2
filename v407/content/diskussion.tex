\section{Diskussion}
\label{sec:Diskussion}

Aus der ersten Messung ergab sich für den Brechungsindex $n_{\perp} = 4.8 ± 8.9$.
Es fällt sofort auf, dass die Standartabweichung immens groß ist.
Dies deutet auf einen oder mehrere große systematische Fehler hin.
Ein Fehler ist sicherlich die mit einer großen Wahrscheinlichkeit falsch aufgenommene Referenzintensität.
Diese liegt laut Originaldaten bei $I_r = 1,4$.
Diese wurde bereits in \autoref{sec:Auswertung} als deutlich höher, nämlich $I_r = 1,67$ angenommen.
Dies liegt daran, dass die Messergebnisse sonst keinen Sinn machen.
Schließlich wären viele Werte dann höher als die Referenz, welches physikalisch nicht möglich ist.
Deshalb wurde eine realistische Referenz gewählt.
Weiterhin war es eine große Schwierigkeit immer genau die Öffnung der Photozelle zu treffen.
Bereits kleine Auslenkungen im Millimeter-Bereich sorgten für große Schwankungen. 
Außerdem wurde die Messung dadurch erschwert, dass die Apparatur wackelte und sich sogar im Verlauf der Experiments löste,
weshalb die Werte teilweise unerwartete Sprünge machten.
Aus diesen Gründen kann die erste Messung nicht als Erfolg angesehen werden, welches auch wie erwähnt leicht durch die Standartabweichung eingesehen werden kann. \\

Die zweite Messung jedoch lässt sich als Erfolg werten, da hier die Standartabweichung nicht so immens groß ist und außerdem
sehr nah an dem Wert liegt, die der Brewsterwinkel liefert.
Dabei ergeben sich $n = 3.2 \pm 1.4$ und $n_\text{Brew} = 3.73$.
Gemittelt ergibt sich hier $\bar{n} = 3.5 \pm 0.7$.
Der Theoriewert von Silizium ist nach Literatur \cite{brechungsindex} bei $n_\text{Lit} = 3.673$.
Damit ergibt sich eine Abweichung von
\begin{equation*}
    \increment n = \frac{3.673 - 3.2}{3.2} \cdot 100 = 14.78 \% 
\end{equation*}
für den Wert aus der Messung der parallelen Polarisation und für den geschätzten Brewsterwinkel
\begin{equation}
    \increment n = \frac{3.73 - 3.673}{3.673} \cdot 100 = 1.55 \% .
\end{equation}
Für den Mittelwert ergibt sich die Abweichung
\begin{equation*}
    \increment n = \frac{3.673 - 3.5}{3.5} \cdot 100 = 4.94 \% .
\end{equation*}
Dies kann als Erfolg gewertet werden, da die zuvor genannten Fehlerquellen hier immer noch beitragen, jedoch die
Messpräzision erhöht wurde.
