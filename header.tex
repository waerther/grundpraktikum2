\documentclass[
  bibliography=totoc,     % Literatur im Inhaltsverzeichnis
  captions=tableheading,  % Tabellenüberschriften
  titlepage=firstiscover, % Titelseite ist Deckblatt
  parskip=half, % !!! halbzeiliger vertikaler Abstand (siehe Latex-Skript S.125)
]{scrartcl}

% !!! Python in Latex
\usepackage{pythontex}

% !!! zum Drehen von Seiten
\usepackage{adjustbox}

% Paket float verbessern
\usepackage{scrhack}

% Warnung, falls nochmal kompiliert werden muss
\usepackage[aux]{rerunfilecheck}

% unverzichtbare Mathe-Befehle
\usepackage{amsmath}
% viele Mathe-Symbole
\usepackage{amssymb}
% Erweiterungen für amsmath
\usepackage{mathtools}

% Fonteinstellungen
\usepackage{fontspec}
% Latin Modern Fonts werden automatisch geladen
% Alternativ zum Beispiel:
%\setromanfont{Libertinus Serif}
%\setsansfont{Libertinus Sans}
%\setmonofont{Libertinus Mono}

% Wenn man andere Schriftarten gesetzt hat,
% sollte man das Seiten-Layout neu berechnen lassen
\recalctypearea{}

% deutsche Spracheinstellungen
\usepackage[ngerman]{babel}

% !!! babel mit anderen Sprachen laden für \enquote (siehe latex-Skript S.33)
\usepackage[autostyle]{csquotes}

% !!! zum durchstreichen durch \cancel{}
\usepackage{amsmath}
\usepackage[makeroom]{cancel}

% !!! zum Ersetzen von \symup{} durch z.B. \dif{}, siehe latex-Skript
\usepackage{expl3}
\usepackage{xparse}
\ExplSyntaxOn
\NewDocumentCommand \dif {m} {
  \mathinner{\symup{d} #1}
}
\ExplSyntaxOff

% !!! Nummerierung von Gleichungen nach Sections: Bei langen Dokumenten empfohlen (Chiristan)
% \usepackage{amsmath}
% \numberwithin{equation}{section}

% % !!! for den "token not defined in pdf"-Fehler, wenn man latex in \section{...} verwendet
% \PassOptionsToPackage{unicode}{hyperref}
% \PassOptionsToPackage{naturalnames}{hyperref} klappt nicht??


\usepackage[
  math-style=ISO,    % ┐
  bold-style=ISO,    % │
  sans-style=italic, % │ ISO-Standard folgen
  nabla=upright,     % │
  partial=upright,   % ┘
  warnings-off={           % ┐
    mathtools-colon,       % │ unnötige Warnungen ausschalten
    mathtools-overbracket, % │
  },                       % ┘
]{unicode-math}

% traditionelle Fonts für Mathematik
\setmathfont{Latin Modern Math}
% Alternativ zum Beispiel:
%\setmathfont{Libertinus Math}

\setmathfont{XITS Math}[range={scr, bfscr}]
\setmathfont{XITS Math}[range={cal, bfcal}, StylisticSet=1]

% Zahlen und Einheiten
\usepackage[
  locale=DE,                   % deutsche Einstellungen
  separate-uncertainty=true,   % immer Unsicherheit mit \pm
  per-mode=symbol-or-fraction, % / in inline math, fraction in display math
]{siunitx}

% chemische Formeln
\usepackage[
  version=4,
  math-greek=default, % ┐ mit unicode-math zusammenarbeiten
  text-greek=default, % ┘
]{mhchem}

% richtige Anführungszeichen
\usepackage[autostyle]{csquotes}

% schöne Brüche im Text
\usepackage{xfrac}

% Standardplatzierung für Floats einstellen
\usepackage{float}
\floatplacement{figure}{htbp}
\floatplacement{table}{htbp}

% Floats innerhalb einer Section halten
\usepackage[
  section, % Floats innerhalb der Section halten
  below,   % unterhalb der Section aber auf der selben Seite ist ok
]{placeins}

% Seite drehen für breite Tabellen: landscape Umgebung
\usepackage{pdflscape}

% Captions schöner machen.
\usepackage[
  labelfont=bf,        % Tabelle x: Abbildung y: ist jetzt fett
  font=small,          % Schrift etwas kleiner als Dokument
  width=0.9\textwidth, % maximale Breite einer Caption schmaler
]{caption}

% subfigure, subtable, subref
\usepackage{subcaption}

% Grafiken können eingebunden werden !!! von Jonas
\usepackage{graphicx}
%\usepackage{wrapfig}%Textumflossene Grafik
\usepackage{cancel} %Brüche Kürzen
\usepackage{tikz} %Fancy Kreisnummern
\usepackage[a4paper, left=30mm, top=30mm, right=30mm, bottom=40mm]{geometry} %Pagelayout
%\usepackage{subcaption} %Unterüberschrift
%\captionsetup[figure]{calcwidth=.85\linewidth} %You dont want to know
\usepackage{tasks} % Aufgaben

% !!! Für das Einfügen von pdfs wie messdaten etc.
\usepackage{pdfpages}

% schöne Tabellen
\usepackage{booktabs}

% Verbesserungen am Schriftbild
\usepackage{microtype}

% Literaturverzeichnis
\usepackage[
  backend=biber,
]{biblatex}
% Quellendatenbank
\addbibresource{lit.bib}
\addbibresource{programme.bib}

% Hyperlinks im Dokument
\usepackage[
  german,
  unicode,        % Unicode in PDF-Attributen erlauben
  pdfusetitle,    % Titel, Autoren und Datum als PDF-Attribute
  pdfcreator={},  % ┐ PDF-Attribute säubern
  pdfproducer={}, % ┘
]{hyperref}
% erweiterte Bookmarks im PDF
\usepackage{bookmark}

% Hinzugefügt
\usepackage{wrapfig}

% Trennung von Wörtern mit Strichen
\usepackage[shortcuts]{extdash}

% % !!! Seitenlayout (Jonas)
% \usepackage[headsepline=1pt,footsepline=1pt]{scrlayer-scrpage}
% \pagestyle{scrheadings}
% \clearpairofpagestyles

\author{%
  Toby Teasdale\\%
  \href{mailto:toby.teasdale@tu-dortmund.de}{toby.teasdale@tu-dortmund.de}%
  \and%
  Erich Wagner\\%
  \href{mailto:erich.wagner@tu-dortmund.de}{erich.wagner@tu-dortmund.de}%
}
\publishers{TU Dortmund – Fakultät Physik}
\setlength\parindent{0pt}