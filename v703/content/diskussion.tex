\section{Diskussion}
\label{sec:Diskussion}

Im Plateaubereich von $380 - \qty{620}{V}$ ergibt sich schließlich die für das Zählrohr 
\begin{equation*}
    \text{charakteristische Steigung}
    M = \qty{1.26(25)}{\percent} \; \text{pro} \; \qty{100}{V} \, .
\end{equation*}

In \autoref{fig:plot1} wird zudem deutlich, dass die Betriebsspannung unterhalb von \qty{700}{V}
zu halten ist, damit der Entladungsbereich des Zählrohrs nicht erreicht wird und es beschädigt.

Die beiden Totzeiten
\begin{align*}
    T_\text{Quell} &\approx \qty{300(400)}{\micro\second} \quad \text{und} \\
    T_\text{Osz} &\approx \qty{240}{\micro\second} \, .
\end{align*}
weichen \qty{20}{\percent} von einander ab. 
Beide Totzeiten liegen in dem typischen Bereich von $10^{4} - 10^{-5} \, \unit{\micro\second}$ eines Geiger-Müller-Zählrohrs.

In \autoref{fig:plot2} wird ein proportionaler Zusammenhang zwischen
freigesetzten Ladungen $Z$ pro einfallendem Teilchen $n$ und gemessenem Strom $I$ deutlich.