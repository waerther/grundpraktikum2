\section{Durchführung}
\label{sec:Durchführung}
\subsection{Technische Daten}
Im folgenden werden zunächst die technischen Daten der einzelnen Komponenten etc aufgelistet:
\begin{align*}
    \textbf{Dopplerphantomflussigkeit:} && \rho &= 1,15  \unit\gram / \unit{\cubic\centi\meter} & \text{Dichte} \\
    && c_L &= 1800 \unit\meter / \unit\second & \text{Schallgeschwindigkeit} \\
    && \nu &= 12 \unit{\milli\pascal\second} & \text{Viskosität} \\
    \\
    \textbf{Dopplerprisma:} && c_P &= 2700 \unit\meter / \unit\second & \text{Schallgeschwindigkeit} \\
    && l &= 30,7 \unit{\milli\meter} & \text{Länge der Vorlaufstrecke} \\
    \\
    \textbf{Strömungsrohre:} && \text{Innendurchmesser} && \text{Außendurchmesser} \\
    && 7 \unit{\milli\meter} && 10 \unit{\milli\meter} \\
    && 10 \unit{\milli\meter} && 15 \unit{\milli\meter} \\
    && 16 \unit{\milli\meter} && 20 \unit{\milli\meter} \\
\end{align*}

\subsection{Vorbereitungsaufgabe}
Die Dopplerwinkel berechnen sich nach der Formel
\begin{equation}
    \alpha = 90° - \text{arcsin}\left(\text{sin} \theta \frac{c_L}{c_P}\right).
\end{equation}
Daraus folgt für
\begin{align*}
    \theta_1 &= 15° &\implies&& \alpha_1 &= 80,06°, \\
    \theta_2 &= 30° &\implies&& \alpha_2 &= 70,53°, \\
    \theta_3 &= 60° &\implies&& \alpha_3 &= 54,74°. \\
\end{align*}
\subsection{Aufbau}
