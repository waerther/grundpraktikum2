\section{Diskussion}
\label{sec:Diskussion}

\subsection{Bestimmung der Strömungsgeschwindigkeit}

In \autoref{fig:winkel} fällt zunächst ein linearer Zusammenhang zwischen Frequenzverschiebung und Strömungsgeschwindigkeit auf.
Die Strömungsgeschwindigkeit nimmt wie erwartet mit höherer Leistung zu.
Die negative Geschwindigkeit bei $\theta = \qty{30}{°}$ liegt aufgrund des Messwinkels vor.


\subsection{Bestimmung des Strömungsprofils}

Bei einem Vergleich der Plots in \autoref{fig:profile} bezüglich der Streuintensität, fällt auf,
dass diese sich relativ gering von einander unterscheiden. Zudem fällt auf, dass die
Intensität und Geschwindigkeit zur Mitte des Rohres zunimmt und und zu den Rändern abfällt.
Die Messtiefen, an denen sich scheinbar keine Geschwindigkeit messen lässt, obwohl sie sich innerhalb des Rohrs befindet
lassen sich durch einige der folgenden Fehlerquellen erklären.

\subsection{Fehlerquellen}

Bei der Durchführung fallen allgemeine Fehlerquellen auf, die durch die Empfindlichkeit der Ultraschallsonde bedingt sind.
So muss darauf geachtet werden, dass sich ausreichend Ultraschall-Gel zwischen der Sonde und dem Prisma, 
sowie zwischen dem Prisma und dem Strömungsrohr befindet, da sonst Luftzwischenräume entstehen und die Daten verfälscht werden können.
Außerdem muss die Ultraschallsonde still gehalten werden, da die gemessenen Werte während der Messung stark schwanken.